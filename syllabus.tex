% Options for packages loaded elsewhere
\PassOptionsToPackage{unicode}{hyperref}
\PassOptionsToPackage{hyphens}{url}
\PassOptionsToPackage{dvipsnames,svgnames,x11names}{xcolor}
%
\documentclass[
  letterpaper,
  DIV=11,
  numbers=noendperiod]{scrartcl}

\usepackage{amsmath,amssymb}
\usepackage{lmodern}
\usepackage{iftex}
\ifPDFTeX
  \usepackage[T1]{fontenc}
  \usepackage[utf8]{inputenc}
  \usepackage{textcomp} % provide euro and other symbols
\else % if luatex or xetex
  \usepackage{unicode-math}
  \defaultfontfeatures{Scale=MatchLowercase}
  \defaultfontfeatures[\rmfamily]{Ligatures=TeX,Scale=1}
\fi
% Use upquote if available, for straight quotes in verbatim environments
\IfFileExists{upquote.sty}{\usepackage{upquote}}{}
\IfFileExists{microtype.sty}{% use microtype if available
  \usepackage[]{microtype}
  \UseMicrotypeSet[protrusion]{basicmath} % disable protrusion for tt fonts
}{}
\makeatletter
\@ifundefined{KOMAClassName}{% if non-KOMA class
  \IfFileExists{parskip.sty}{%
    \usepackage{parskip}
  }{% else
    \setlength{\parindent}{0pt}
    \setlength{\parskip}{6pt plus 2pt minus 1pt}}
}{% if KOMA class
  \KOMAoptions{parskip=half}}
\makeatother
\usepackage{xcolor}
\setlength{\emergencystretch}{3em} % prevent overfull lines
\setcounter{secnumdepth}{-\maxdimen} % remove section numbering
% Make \paragraph and \subparagraph free-standing
\ifx\paragraph\undefined\else
  \let\oldparagraph\paragraph
  \renewcommand{\paragraph}[1]{\oldparagraph{#1}\mbox{}}
\fi
\ifx\subparagraph\undefined\else
  \let\oldsubparagraph\subparagraph
  \renewcommand{\subparagraph}[1]{\oldsubparagraph{#1}\mbox{}}
\fi


\providecommand{\tightlist}{%
  \setlength{\itemsep}{0pt}\setlength{\parskip}{0pt}}\usepackage{longtable,booktabs,array}
\usepackage{calc} % for calculating minipage widths
% Correct order of tables after \paragraph or \subparagraph
\usepackage{etoolbox}
\makeatletter
\patchcmd\longtable{\par}{\if@noskipsec\mbox{}\fi\par}{}{}
\makeatother
% Allow footnotes in longtable head/foot
\IfFileExists{footnotehyper.sty}{\usepackage{footnotehyper}}{\usepackage{footnote}}
\makesavenoteenv{longtable}
\usepackage{graphicx}
\makeatletter
\def\maxwidth{\ifdim\Gin@nat@width>\linewidth\linewidth\else\Gin@nat@width\fi}
\def\maxheight{\ifdim\Gin@nat@height>\textheight\textheight\else\Gin@nat@height\fi}
\makeatother
% Scale images if necessary, so that they will not overflow the page
% margins by default, and it is still possible to overwrite the defaults
% using explicit options in \includegraphics[width, height, ...]{}
\setkeys{Gin}{width=\maxwidth,height=\maxheight,keepaspectratio}
% Set default figure placement to htbp
\makeatletter
\def\fps@figure{htbp}
\makeatother

\KOMAoption{captions}{tableheading}
\makeatletter
\@ifpackageloaded{tcolorbox}{}{\usepackage[many]{tcolorbox}}
\@ifpackageloaded{fontawesome5}{}{\usepackage{fontawesome5}}
\definecolor{quarto-callout-color}{HTML}{909090}
\definecolor{quarto-callout-note-color}{HTML}{0758E5}
\definecolor{quarto-callout-important-color}{HTML}{CC1914}
\definecolor{quarto-callout-warning-color}{HTML}{EB9113}
\definecolor{quarto-callout-tip-color}{HTML}{00A047}
\definecolor{quarto-callout-caution-color}{HTML}{FC5300}
\definecolor{quarto-callout-color-frame}{HTML}{acacac}
\definecolor{quarto-callout-note-color-frame}{HTML}{4582ec}
\definecolor{quarto-callout-important-color-frame}{HTML}{d9534f}
\definecolor{quarto-callout-warning-color-frame}{HTML}{f0ad4e}
\definecolor{quarto-callout-tip-color-frame}{HTML}{02b875}
\definecolor{quarto-callout-caution-color-frame}{HTML}{fd7e14}
\makeatother
\makeatletter
\makeatother
\makeatletter
\makeatother
\makeatletter
\@ifpackageloaded{caption}{}{\usepackage{caption}}
\AtBeginDocument{%
\ifdefined\contentsname
  \renewcommand*\contentsname{Table of contents}
\else
  \newcommand\contentsname{Table of contents}
\fi
\ifdefined\listfigurename
  \renewcommand*\listfigurename{List of Figures}
\else
  \newcommand\listfigurename{List of Figures}
\fi
\ifdefined\listtablename
  \renewcommand*\listtablename{List of Tables}
\else
  \newcommand\listtablename{List of Tables}
\fi
\ifdefined\figurename
  \renewcommand*\figurename{Figure}
\else
  \newcommand\figurename{Figure}
\fi
\ifdefined\tablename
  \renewcommand*\tablename{Table}
\else
  \newcommand\tablename{Table}
\fi
}
\@ifpackageloaded{float}{}{\usepackage{float}}
\floatstyle{ruled}
\@ifundefined{c@chapter}{\newfloat{codelisting}{h}{lop}}{\newfloat{codelisting}{h}{lop}[chapter]}
\floatname{codelisting}{Listing}
\newcommand*\listoflistings{\listof{codelisting}{List of Listings}}
\makeatother
\makeatletter
\@ifpackageloaded{caption}{}{\usepackage{caption}}
\@ifpackageloaded{subcaption}{}{\usepackage{subcaption}}
\makeatother
\makeatletter
\@ifpackageloaded{tcolorbox}{}{\usepackage[many]{tcolorbox}}
\makeatother
\makeatletter
\@ifundefined{shadecolor}{\definecolor{shadecolor}{rgb}{.97, .97, .97}}
\makeatother
\makeatletter
\makeatother
\ifLuaTeX
  \usepackage{selnolig}  % disable illegal ligatures
\fi
\IfFileExists{bookmark.sty}{\usepackage{bookmark}}{\usepackage{hyperref}}
\IfFileExists{xurl.sty}{\usepackage{xurl}}{} % add URL line breaks if available
\urlstyle{same} % disable monospaced font for URLs
\hypersetup{
  pdftitle={STA 360: Bayesian methods and modern statistics},
  colorlinks=true,
  linkcolor={blue},
  filecolor={Maroon},
  citecolor={Blue},
  urlcolor={Blue},
  pdfcreator={LaTeX via pandoc}}

\title{STA 360: Bayesian methods and modern statistics}
\usepackage{etoolbox}
\makeatletter
\providecommand{\subtitle}[1]{% add subtitle to \maketitle
  \apptocmd{\@title}{\par {\large #1 \par}}{}{}
}
\makeatother
\subtitle{Spring 2025}
\author{}
\date{}

\begin{document}
\maketitle
\ifdefined\Shaded\renewenvironment{Shaded}{\begin{tcolorbox}[interior hidden, boxrule=0pt, sharp corners, breakable, borderline west={3pt}{0pt}{shadecolor}, enhanced, frame hidden]}{\end{tcolorbox}}\fi

\hypertarget{syllabus}{%
\section{Syllabus}\label{syllabus}}

\hypertarget{course-description}{%
\subsubsection{Course description}\label{course-description}}

This course introduces Bayesian modeling and inference, motivated by
real world examples. Course topics include Bayes' theorem,
exchangeability, conjugate priors, Markov chain Monte Carlo (MCMC),
Gibbs sampling, Metropolis-Hastings, hierarchical modeling, Bayesian
regression and generalized linear models. We compare and contrast
Bayesian methods to the frequentist paradigm. By the end of this course
students should feel comfortable (1) writing Bayesian models and, when
appropriate, (2) sampling from the posterior using MCMC to make
inference.

\hypertarget{logistics}{%
\subsubsection{Logistics}\label{logistics}}

\hypertarget{teaching-team-office-hours}{%
\paragraph{Teaching team \& office
hours}\label{teaching-team-office-hours}}

\begin{longtable}[]{@{}
  >{\raggedright\arraybackslash}p{(\columnwidth - 6\tabcolsep) * \real{0.2500}}
  >{\raggedright\arraybackslash}p{(\columnwidth - 6\tabcolsep) * \real{0.2500}}
  >{\raggedright\arraybackslash}p{(\columnwidth - 6\tabcolsep) * \real{0.2500}}
  >{\raggedright\arraybackslash}p{(\columnwidth - 6\tabcolsep) * \real{0.2500}}@{}}
\toprule()
\begin{minipage}[b]{\linewidth}\raggedright
\end{minipage} & \begin{minipage}[b]{\linewidth}\raggedright
Contact
\end{minipage} & \begin{minipage}[b]{\linewidth}\raggedright
Office hours
\end{minipage} & \begin{minipage}[b]{\linewidth}\raggedright
Location
\end{minipage} \\
\midrule()
\endhead
Dr.~Alexander Fisher & \url{aaf29@duke.edu} & We/Th/Fr: 10:00-11:00am &
Old Chem 223B \\
Matt O'Donnell & \url{matthew.l.odonnell@duke.edu} & TBD & Old Chem
203B \\
Sophia Yang & \url{sophia.yang@duke.edu} & TBD & Old Chem 203B \\
Yihao Gu & \url{yihao.gu@duke.edu} & TBD & Old Chem 203B \\
Yueqi Guo & \url{yueqi.guo@duke.edu} & TBD & Old Chem 203B \\
\bottomrule()
\end{longtable}

\hypertarget{meetings}{%
\paragraph{Meetings}\label{meetings}}

\begin{longtable}[]{@{}lll@{}}
\toprule()
\endhead
Lecture & We/Fr 11:45am - 1:00pm & Reuben-Cooke Building 130 \\
Lab 01 & Th 1:25pm - 2:40pm & Old Chemistry 201 \\
Lab 02 & Th 3:05pm - 4:20pm & Old Chemistry 201 \\
\bottomrule()
\end{longtable}

Course website:
\href{https://sta360-fa24.github.io/}{sta360-fa24.github.io}

\hypertarget{course-material}{%
\subsubsection{Course material}\label{course-material}}

\begin{itemize}
\item
  \href{https://pdhoff.github.io/book/}{A First Course in Bayesian
  Statistical Methods}. As a Duke student, an electronic version of the
  book is freely available to you on Springer link. Check the errata at
  the link above.
\item
  \href{/chapterSummaries.html}{Chapter summaries}. I compile major
  take-away points from each section. Review these to help prepare for
  exams.
\item
  We will use the statistical software package R on homework asignments
  in this course. R is freely available at
  \url{http://www.r-project.org/}. RStudio, the popular IDE for R, is
  freely available at \url{https://posit.co/downloads/}.
\end{itemize}

\hypertarget{schedule-of-topics}{%
\subsubsection{Schedule of topics}\label{schedule-of-topics}}

Part I: The Bayesian modeling toolkit

\begin{enumerate}
\def\labelenumi{\arabic{enumi}.}
\tightlist
\item
  Review of probability
\item
  Conjugate statistical models
\item
  Posterior summaries and Monte Carlo sampling
\item
  Markov chain Monte Carlo (Metropolis-Hastings)
\end{enumerate}

Part II: Statistical model building and analysis

\begin{enumerate}
\def\labelenumi{\arabic{enumi}.}
\tightlist
\item
  Semi-conjugate models and Gibbs sampling
\item
  Linear regression
\item
  Generalized linear models
\item
  Hierarchical models
\end{enumerate}

\newpage

\hypertarget{evaluation}{%
\subsubsection{Evaluation}\label{evaluation}}

\begin{longtable}[]{@{}
  >{\raggedright\arraybackslash}p{(\columnwidth - 2\tabcolsep) * \real{0.3684}}
  >{\raggedright\arraybackslash}p{(\columnwidth - 2\tabcolsep) * \real{0.6316}}@{}}
\toprule()
\begin{minipage}[b]{\linewidth}\raggedright
Assignment
\end{minipage} & \begin{minipage}[b]{\linewidth}\raggedright
Description
\end{minipage} \\
\midrule()
\endhead
Homework (40\%) & Individual take-home assignments, submitted to
Gradescope. \\
Midterms (30\%) & Two in-class exams. \\
Final exam (25\%) & Cumulative final during final's week. \\
Quizzes (5\%) & In-class pop quizzes. \\
\bottomrule()
\end{longtable}

A \(>= 93\), A- \(< 93\), B+ \(< 90\), B \(< 87\), B- \(< 83\), C+
\(<80\), C \(< 77\), C- \(< 73\), D+ \(< 70\), D \(< 67\), D- \(< 63\),
F \(< 60\)

\begin{tcolorbox}[enhanced jigsaw, opacityback=0, bottomrule=.15mm, titlerule=0mm, colbacktitle=quarto-callout-note-color!10!white, breakable, leftrule=.75mm, toprule=.15mm, rightrule=.15mm, left=2mm, bottomtitle=1mm, colback=white, title=\textcolor{quarto-callout-note-color}{\faInfo}\hspace{0.5em}{A note on quizzes}, arc=.35mm, toptitle=1mm, coltitle=black, opacitybacktitle=0.6, colframe=quarto-callout-note-color-frame]
On random class days, there will be a brief quiz on the previous
lectures. If you score \(>60\%\) cumulatively on your final quiz grade,
you will receive full participation credit. Your lowest \textbf{two}
quizzes will also be dropped.
\end{tcolorbox}

\begin{tcolorbox}[enhanced jigsaw, opacityback=0, bottomrule=.15mm, titlerule=0mm, colbacktitle=quarto-callout-note-color!10!white, breakable, leftrule=.75mm, toprule=.15mm, rightrule=.15mm, left=2mm, bottomtitle=1mm, colback=white, title=\textcolor{quarto-callout-note-color}{\faInfo}\hspace{0.5em}{A note on exams}, arc=.35mm, toptitle=1mm, coltitle=black, opacitybacktitle=0.6, colframe=quarto-callout-note-color-frame]
If you miss either midterm 1 or midterm 2, \textbf{and have an excused
absence}, your missing midterm grade will be replaced by your final exam
grade. You must take at least 1 midterm and the final exam to pass the
course.
\end{tcolorbox}

\hypertarget{policies}{%
\subsubsection{Policies}\label{policies}}

\textbf{Academic integrity}

By enrolling in this course, you commit to upholding Duke's community
standard reproduced as follows:

\begin{quote}
I will not lie, cheat, or steal in my academic endeavors;

I will conduct myself honorably in all my endeavors; and

I will act if the Standard is compromised.
\end{quote}

Any violations of academic integrity will automatically result in a 0
for the assignment and will be reported to the Office of Student Conduct
for further action. For the Exams and Quizzes, students are required to
work alone. For the Homework assignments, students may work with a study
group but each student must write up and submit their own answers.

\newpage

\textbf{Late work}

Late homework may be submitted within 48 hours of the assignment
deadline. Late homework submitted within 24 hours (even 1 minute late)
will receive a 5\% late penalty. Late work submitted between 24 to 48
hours of the deadline will receive a 10\% late penalty. Work submitted
after 48 hours will not be accepted. Exams cannot be turned in late and
can only be excused under exceptional circumstances. The Duke policy for
illness requires a short-term illness report or a letter from the Dean;
except in emergencies, all other absenteeism must be approved in advance
(e.g., an athlete who must miss class may be excused by prior
arrangement for specific days). For emergencies, email notification is
needed at the first reasonable time.

\textbf{Errors in grading}

Errors in grading must be brought to the attention of the TA or
instructor during office hours within 1 week of receiving the grade.



\end{document}
